\documentclass{article}

\usepackage[utf8]{inputenc}
\usepackage[T1]{fontenc}
\usepackage[brazil]{babel}
\usepackage{sbc-template}

\title{Protótipo de \emph{Interface} Humano Computador para Supermercados
Utilizando Técnicas de Lojas Virtuais e Computação Ubíqua}
\author{Wanderson Henrique Camargo Rosa \and Roberto Raguze Flores}
\address{Centro de Ciências Exatas e Tecnológicas\\Universidade do Vale do Rio
dos Sinos (UNISINOS)\email{\{wandersonwhcr \and roberto.raguze\}@gmail.com}}

\begin{document}

\maketitle{}

\section{Introdução}

% Contextualização Geral

A utilização de ambientes virtuais para compras através de lojas eletrônicas
está cada vez mais comum nos dias de hoje. Através deste método de venda é
possível atingir mais facilmente o público alvo com promoções, dicas de produtos
semelhantes e comparação de preços.

% Negativos de Lojas Virtuais

Porém estes ambientes não fornecem ao cliente a principal característica de
venda de um produto: o contato direto. O ato do contato físico com produto pode
ser a ação chave para a compra do mesmo ou escolha de outros itens semelhantes,
algo que somente um comércio tradicional pode nos fornecer.

Se por lado a informatização nos traz a facilidade de comunicar as pessoas sobre
o produto em questão, a venda tradicional pode nos dar uma melhor segurança
durante a compra. Porém, durante a compra comum, clientes enfrentam a falta de
informações de um determinado produto, recorrendo à pesquisa de dados adicionais
em lojas virtuais \cite{vonreischach2009}.

% União das Idéias

Dado que ambos os lados possuem características interessantes, podemos
desenvolver uma \emph{interface} para uma loja comum, trazendo as facilidades
virtuais para o mundo real, através de ambientes de realidade aumentada.

\subsection{Interação Humano Computador}

% Realidade

Em um estabelecimento real, um produto fica disponível ao consumidor, podendo
este interagir com o objeto. Informações adicionais podem ser fornecidas por
funcionários da loja, contudo, nem sempre a pessoa responsável pela venda
conhece completamente todos os produtos da loja ou pode fornecer dados técnicos
especializados, muito menos sugestões de todos os clientes que já compraram o
objeto.

% Realidade Aumentada

Por meio de dispositivos móveis, o cliente poderá acessar informações
pertinentes a qualquer produto da loja, pesquisar sobre produtos disponíveis e
em promoção, bem como controlar sua compra atual.

\subsection{Idealização}

Com esta análise de informações, há uma proposta de criação de uma
\emph{interface} para supermercados que utilize técnicas de lojas virtuais,
tornando a compra mais fácil e controlada, através de dispositivos ubíquos.

\section{Protótipo}

\section{Conclusão}

\bibliographystyle{sbc}
\bibliography{sbc-template}

\end{document}