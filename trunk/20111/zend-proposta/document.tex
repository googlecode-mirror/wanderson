\documentclass{article}

% Configurações Genéricas ------------------------------------------------------
\usepackage[utf8]{inputenc}
\usepackage[T1]{fontenc}
\usepackage[brazil]{babel}
\usepackage{sbc-template}

% Hifenização ------------------------------------------------------------------
\hyphenation{frame-works}

% Informações Pessoais ---------------------------------------------------------
\title{Proposta para Introdução ao Zend Framework}
\author{Wanderson Henrique Camargo Rosa}
\address{Centro de Ciências Exatas e Tecnológicas\\Universidade do Vale do Rio
dos Sinos --- UNISINOS\email{wandersonwhcr@gmail.com}}

% Documento --------------------------------------------------------------------
\begin{document}

% Título -----------------------------------------------------------------------
\maketitle{}

% Apresentação -----------------------------------------------------------------
\section{Apresentação}
\label{sec:apresentacao}

Para auxiliar no desenvolvimento de aplicativos específicos utilizando
determinadas linguagens de programação, podemos recorrer a bibliotecas
específicas chamadas \emph{frameworks}. Como muitas linguagens, o PHP possui
muitos \emph{frameworks} que facilitam a programação, com estruturas prontas e
regras específicas, padronizando o desenvolvimento e otimizando o tempo de
produção do aplicativo.

Estendendo a arte e o espírito do PHP, o Zend Framework é uma biblioteca
baseada em simplicidade, melhores práticas de orientação a objetos, livre
licença corporativa e testado rigorosamente \cite{zend}. Possuindo uma
arquitetura flexível e componentes com baixa dependência entre si, podemos
utilizá-lo em qualquer projeto da linguagem buscando usufruir de códigos seguros
e de fácil reaproveitamento.

Este texto apresenta uma proposta de Introdução ao Zend Framework, que deverá
ser ministrado durante a Semana Acadêmica de Informática desta instituição. O
encontro de exibição, terá carga horária de 2 (duas) horas e abordagem sobre
utilização da biblioteca e criação de um simples projeto, envolvendo os
principais padrões de projetos disponíveis.

% Bibliografia -----------------------------------------------------------------
\bibliographystyle{sbc}
\bibliography{document}

\end{document}