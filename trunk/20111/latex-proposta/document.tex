\documentclass{article}

% Configurações Genéricas ------------------------------------------------------
\usepackage[utf8]{inputenc}
\usepackage[T1]{fontenc}
\usepackage[brazil]{babel}
\usepackage{sbc-template}

% Informações Pessoais ---------------------------------------------------------
\title{Proposta para Minicurso de Introdução ao LaTeX}
\author{Wanderson Henrique Camargo Rosa}
\address{Centro de Ciências Exatas e Tecnológicas\\Universidade do Vale do Rio
dos Sinos --- UNISINOS\email{wandersonwhcr@gmail.com}}

% Documento --------------------------------------------------------------------
\begin{document}

% Título -----------------------------------------------------------------------
\maketitle{}

% Apresentação -----------------------------------------------------------------
\section{Apresentação}
\label{sec:apresentacao}

A construção de documentos a partir de normas estabelecidas é dificultada quando
há intervenção humana, conceptível ao erro. Com a possibilidade de utilização de
um motor para formatação mecânica a partir de marcações em texto, preocupando-se
somente com a lógica do conteúdo, o aluno pode estruturar seus artigos e
trabalhos, deixando a responsabilidade de visualização para terceiros, função do
sistema tipográfico \LaTeX{}.

Este texto apresenta uma Proposta de Minicurso de Introdução ao \LaTeX{}, que
será ministrado durante a Semana Acadêmica de Informática desta instituição. O
encontro de exibição, doravante chamado \emph{Workshop}, terá carga horária de 2
(duas) horas e abordagem acadêmica superficial sobre história, vantagens,
aplicações, construção e conteúdo de pesquisa sobre o sistema tipográfico
\LaTeX{} para criação de documentos.

% Objetivos --------------------------------------------------------------------
\section{Objetivos}
\label{sec:objetivos}

Criar um artigo que esteja padronizado conforme as normas da Sociedade
Brasileira de Computação (SBC), disponibilizadas em pacote específico por aquela
instituição, utilizando \LaTeXe{} e ferramentas de código aberto. Aplicar
referências bibliográficas utilizando Bib\TeX{}.

% Público Alvo -----------------------------------------------------------------
\section{Público Alvo}
\label{sec:publico-alvo}

Pessoas iniciantes que buscam criar a habilidade de escrever textos no formato
\LaTeX{}. Pessoas que necessitam desenvolver documentos padronizados que
utilizem normas da Associação Brasileira de Normas Técnicas (ABNT) ou artigos
científicos conforme normas da Sociedade Brasileira de Computação (SBC).
Acadêmicos que estão desenvolvendo o Trabalho de Conclusão de Curso (TCC) ou
artigos para conferências, buscando facilitar a padronização do documento.

% Pré-Requisitos ---------------------------------------------------------------
\section{Pré-Requisitos}
\label{sec:pre-requisitos}

Não há pré-requisitos para este \emph{Workshop}, porém é desejável que o aluno
possua conhecimentos em linguagens de marcação de textos como HTML, BBCode ou
sintaxe Mediawiki.

% Conteúdo Programático --------------------------------------------------------
\section{Conteúdo Programático}
\label{sec:conteudo-programatico}

\begin{enumerate}
    \item Definição
    \item Histórico
    \begin{enumerate}
        \item Donald Knuth e \TeX{}
        \item Leslie Lamport e \LaTeX{}
    \end{enumerate}
    \item Vantagens
    \item Ferramentas
    \item Características
    \item Estruturação de Documento
    \begin{enumerate}
        \item Classes de Documentos
        \item Pacotes
        \item Dados de Identificação
        \item Capítulos e Seções
        \item Referências Cruzadas
        \item Notas de Rodapé
        \item Listas Numeradas e Não Numeradas
    \end{enumerate}
    \item Citações Bibliográficas
    \item Figuras
    \item Bibliografia para Pesquisa
\end{enumerate}

% Método de Aplicação ----------------------------------------------------------
\section{Método de Aplicação}
\label{sec:metodo-de-aplicacao}

A aplicação deste \emph{Workshop} deverá ser feita em laboratórios munidos de
computadores com sistema operacional capaz de executar o \LaTeX{},
preferencialmente \emph{*unix-like}\footnote{Sistemas operacionais baseados em
Unix, como os que são executados sobre o \emph{kernel} do Linux} com pacote
Abn\TeX{}, distribuídos em, no máximo, duas pessoas por máquina. Outras formas
podem ser aplicadas com estudo prévio do caso.

Necessidade de instalação de aplicativos para facilitar o desenvolvimento da
aula, capazes de criar a documentação através de um ambiente amigável. É
proposto o uso do \emph{plugin} \TeX{}lipse para Eclipse. Disponibilidade de
visualizador de documentos em formato PDF para exibição dos resultados gerados
durante o \emph{Workshop}.

Aula prática onde o aluno consiga aplicar os conhecimentos exibidos.
Apresentação de \emph{slides} conforme conteúdo programático descrito na Seção
\ref{sec:conteudo-programatico}. Fornecimento de folha para auxílio com dicas e
sugestões de comandos e sintaxe. Será disponibilizado o código fonte da
apresentação para auxiliar os estudos durante a aula.

% Conhecimentos ----------------------------------------------------------------
\section{Conhecimentos}
\label{sec:conhecimentos}

Definir o que é \LaTeX{}. Breve histórico de Donald Knuth e suas contribuições
para a Ciência da Computação. Motivação da criação de motor para formatação de
documentos \cite{knuth1970}. Comentários sobre a complexidade de criação de um
documento que utiliza o motor. Breve histórico de Leslie Lamport e suas
contribuições. Motivação da criação de um sistema tipográfico de documentos que
facilite a construção sobre um motor de formatação \cite{lamport1985}. Vantagens
da utilização de um sistema tipográfico \cite{oetiker2008}. Preocupação com a
estrutura lógica do documento, extraindo formatações. Ferramentas de auxílio.
Ambientes amigáveis de desenvolvimento. Características de documentos, ignorando
arquivos gerados em tempo de execução. Estrutura básica de um documento. Classes
de documentos disponíveis na instalação padrão. Utilização de pacotes
adicionais. Informar dados pessoais de autoria do documento. Estruturar
documentos em seções e exemplificar capítulos. Apontar elementos utilizando
referência cruzada. Construir notas de rodapé. Construir listas numeradas e não
numeradas. Construção de arquivo de referências bibliográficas
\cite{patashnik1988}. Inserir figuras e como criar referências cruzadas conforme
normas necessárias. Exibição de classe para criação de apresentações em
\emph{slides} \cite{tantau2005}. Exibição de livros para pesquisa complementar
\cite{fiorio2005,moses2007,mittelback2004,downes2002}. Informações adicionais e
\emph{sites} para pesquisa.

% Aptidões Adquiridas ----------------------------------------------------------
\section{Aptidões Adquiridas}
\label{sec:aptidoes-adquiridas}

O aluno estará apto a criar um artigo simples, segundo normas da SBC fornecidas
no \emph{site} daquela instituição. Também poderá pesquisar conteúdos sobre
\LaTeX{} em bibliografias exibidas e filtrar informações relevantes durante a
busca. Capacidade de escrever referências bibliográficas a utilizando métodos
auxiliares como o Bib\TeX{}.

% Informações Adicionais -------------------------------------------------------
\section{Informações Adicionais}
\label{sec:informacoes-adicionais}

Solicito que este minicurso possua limite máximo de alunos maior do que a
quantidade de computadores em laboratório. Isto porque no último \emph{Workshop}
ministrado, muitas pessoas não compareceram, ocupando vagas de alunos
interessados na presença. Também existiram aqueles que tentaram cadastro e não
conseguiram por causa do limite máximo de vagas preenchidas no \emph{site}.

% Bibliografia -----------------------------------------------------------------
\bibliographystyle{sbc}
\bibliography{document}

\end{document}