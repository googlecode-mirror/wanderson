\documentclass{article}

% ------------------------------------------------------ Configurações Genéricas
\usepackage[utf8]{inputenc}
\usepackage[T1]{fontenc}
\usepackage[brazil]{babel}
\usepackage{sbc-template}

% --------------------------------------------------------- Informações Pessoais
\title{Tarefa 01 sobre Artigos Disponibilizados}
\author{Wanderson Henrique Camargo Rosa\inst{1}}
\address{Programação para Dispositivos Móveis 2011/1\\Centro de Ciências Exatas
e Tecnológicas\\Universidade do Vale do Rio dos Sinos ---
UNISINOS\email{wandersonwhcr@gmail.com}}

% -------------------------------------------------------------------- Documento
\begin{document}

\maketitle{}

\section{O que um \emph{smartphone} pode fazer hoje e poderá fazer no futuro?}

Hoje, os dispositivos móveis oferecem características nunca possíveis para
interação com informação digital que corresponde aos locais físicos de seus
usuários. Mas estes dispositivos possuem limitados recursos de entrada, muitas
vezes somente teclado e áudio, enquanto muitas formas de interação emergentes
estão começando a tomar vantagem dos movimentos e gestos dos usuários através de
sensores e conteúdo.

\section{Que tipos de intervenções um dispositivo móvel pode permitir?}

Dispositivos móveis são utilizados de diferentes maneiras como \emph{interfaces}
de espaço, incluindo quatro paradigmas de interação principais, característico
dos sistemas MSI.

\subsection{Varinha Mágica}

Um dispositivo que pode ser apontado para objetos distantes e fornece acesso a
mais informações utilizando uma bússola como sensor.

\subsection{Lentes Inteligentes}

Quando colocadas no campo de visão, podemos visualizar conteúdos digitais
diretamente sobre os objetos do mundo real através de realidade aumentada.

\subsection{Olho Mágico Virtual}

Ao invés de fornecer o ambiente atual, o dispositivo pode servir como um olho
mágico para exibir lugares distantes no tempo e espaço.

\subsection{Sexto Sentido}

Dispositivos também podem ser capazes de alertar os usuários de muitas maneiras
sobre mudanças em um ambiente dinâmico.

\section{Como os dispositivos móveis se integram com o ambiente ao nosso redor?}

Um turista pode questionar sobre algum local não familiar, apontando para este
de forma intuitiva utilizando um dispositivo que responde diretamente aos seus
interesses. Quando alguém solicita direcionamentos em viagens, a descrição dos
movimentos podem incluir recursos locais, como características do ambiente
atual. Também pode ser fornecido um ambiente virtual semelhante ao real no
dispositivo, facilitando o reconhecimento pelo usuário. Melhorando, pode-se
informar a distância aproximada conforme rota previamente configurada e
adaptar-se conforme alterações no ambiente. Também podem ser inclusos sinais de
áudio para evitar que o usuário necessite trocar atenção entre a tela e o
ambiente.

\section{O que é sensibilidade espacial?}

Sensibilidade espacial é a habilidade que um dispositivo possui para encontrar
sua orientação sobre o ambiente físico. Isto pode ser formado sobre três
tecnologias:

\subsection{Cálculo Geoespacial}

Atravé de um GPS interno ao dispositivo, uma bússola eletrônica e um
acelerômetro, o campo de visão do usuário poderá ser calculado.

\subsection{Detecção Visual}

A orientação do dispositivo também pode ser detectada através de uma câmera
disponibilizada e algorítmos de visão computacional.

\subsection{Monitoramento em Tempo Real}

Detecta recursos locais em imagens e vídeos capturados, utilizando realidade
aumentada e informações precisas com sobreposição digital.

\section{O que é conteúdo digital georeferenciado?}

A georeferência sobre conteúdos é o ato de gravar informações de posicionamento
global no conteúdo capturado, adicionando uma semântica e associação com
topologias do ambiente.

\section{O que são ambientes 3D?}

Ambientes 3D são características baseadas em escaneamento através de
\emph{lasers} e fotografias aéreas, que podem automaticamente fornecer a criação
de uma modelagem em três dimensões de larga escala de um determinado ambiente.
Estes modelos aplicados ao ambiente urbano fornecem uma nova maneira de
visualizar, organizar e processar o conteúdo sobre espaço.

\section{O que é MSI e quais os desafios?}

MSI, ou Interação Espacial Móvel (Mobile Spatial Interaction) é uma síntese de
muitas formas emergentes de tendência sobre pesquisas, que cobrem novas
interações do usuário de modo físico, natural e ambientação urbana, utilizando
os atuais dispositivos móveis com sensores variados.

\end{document}