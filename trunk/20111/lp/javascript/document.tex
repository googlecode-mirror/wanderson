\documentclass[hyperref={pdfpagelabels=false}]{beamer}

% Configurações Básicas --------------------------------------------------------
\let\Tiny=\tiny
\usepackage[utf8]{inputenc}
\usepackage[T1]{fontenc}
\usepackage[brazil]{babel}

% Informações Pessoais ---------------------------------------------------------
\author[ROSA]{Wanderson Henrique Camargo Rosa}
\title[JS]{Javascript}
\subtitle{Breve Descrição e Histórico da Linguagem}
\institute[UNISINOS]{Linguagens de Programação 2011/1\\Centro de
Ciências Exatas e Tecnológicas\\Universidade do Vale do Rio dos Sinos ---
UNISINOS}

\begin{document}

\begin{frame}
    \maketitle{}
\end{frame}

\begin{frame}{Javascript}{Sobre a Linguagem de Programação}
    Javascript\cite{mozilla,wikipedia} é uma linguagem de script orientada a
    objetos baseada em protótipos, dinâmica, com tipagem fraca e funções de
    primeira classe.
\end{frame}

\begin{frame}{Javascript}{Ambientes de Utilização}
    Basicamente, o Javascript é utilizado principalmente em navegadores
    \emph{web}\cite{mozilla} no cliente, porém podem ser executados em
    servidores\cite{mozilla} que recebem e enviam informações Javascript,
    documentos no formato PDF, navegadores específicos e \emph{widgets} para
    áreas de trabalho\cite{wikipedia}.
\end{frame}

\begin{frame}{Javascript}{Histórico}
    A linguagem de programação Javascript foi criada por Brendan
    Eich\cite{mozilla,wikipedia} na Netscape, sobre um motor chamado
    SpiderMonkey\cite{mozilla}, implementado em C. Primeiramente nomeado como
    Mocha\cite{wikipedia} e logo após
    Livescript\cite{history,diff,wikipedia,plotter}, o Javascript foi idealizado
    para tornar as páginas mais dinâmicas, evitando uma nova requisição para o
    serviço \emph{web}\cite{history}.
\end{frame}

\begin{frame}{Javascript}{Padronização Internacional}
    Javascript recebeu em 1996 uma padronização internacional\cite{history} com
    a \emph{European Computer Manufacturers Association}, sob o nome ECMA-262 ou
    ECMAScript. Existem outras linguagens baseadas com o Javascript, como a
    JScript da Microsoft\cite{diff}. A empresa afirma que o JScript não é
    executável sobre ambientes ECMAScript\cite{wikipedia}.
\end{frame}

\begin{frame}{Javascript}{Programação Orientada a Protótipos}
    Programação baseada em protótipos\cite{prototype} é um estilo de linguagem
    de programação orientada a objetos em que classes não estão presentes. A
    reutilização de código é dada com a clonagem de objetos existentes. Este
    estilo de programação também pode ser chamado de Programação Orientada a
    Protótipos\cite{prototype}.

    Cada objeto possui uma referência para o objeto clonado e não nula,
    fornecendo o encadeamento de protótipos\cite{ecma}.
\end{frame}

\begin{frame}
    \bibliographystyle{plain}
    \bibliography{document}
\end{frame}

\end{document}