\documentclass{article}

% Configurações Genéricas ------------------------------------------------------
\usepackage[utf8]{inputenc}
\usepackage[T1]{fontenc}
\usepackage[brazil]{babel}
\usepackage{sbc-template}
\usepackage{graphicx}

% Informações Pessoais ---------------------------------------------------------
\title{Especificação de Trabalho de Grau B}
\author{Wanderson Henrique Camargo Rosa\inst{1}}
\address{Linguagens de Programação 2011/1\\Centro de Ciências Exatas e
Tecnológicas\\Universidade do Vale do Rio dos Sinos ---
UNISINOS\email{wandersonwhcr@gmail.com}}

\hyphenation{antlrphpruntime postgresql}

% Documento --------------------------------------------------------------------
\begin{document}

\maketitle{}

% Introdução -------------------------------------------------------------------
\section{Introdução}
\label{sec:introducao}

Este documento tem por objetivo especificar brevemente a idéia inicial sobre o
desenvolvimento do Trabalho de Grau B desta disciplina.

Será criado um sistema \textit{web} capaz de gerenciar artigos no formato
disponível pela Sociedade Brasileira de Computação (SBC), escritos utilizando um
subconjunto da linguagem de marcação de texto Wiki Creole. O sistema deverá ser
capaz de traduzir este conteúdo para o formato \LaTeX{}, disponibilizando-o para
o usuário. Também deve ser possível a exportação do conteúdo para arquivo no
formato PDF.

% Motivação --------------------------------------------------------------------
\section{Motivação}
\label{sec:motivLorem ipsumacao}

Documentos técnicos e padronizados geralmente obedecem a regras específicas de
formatação. A comunidade acadêmica, em muitos casos, desconhece ferramentas que
podem facilitar a criação de documentos científicos, abstraindo a formatação dos
mesmos.

O \LaTeX{} é um conjunto de macros e programas que tornam o \TeX{} mais fácil de
ser usado, possuindo muitos modelos e estilos disponíveis para formatos
específicos \cite{p241-salzberg}. \TeX{} é reconhecido pela sua ``curva íngrime
de aprendizado'' e um usuário iniciante adapta-se mais rapidamente se já conhece
alguma linguagem de marcação de textos.

Por este motivo, podemos procurar uma outra linguagem que abstraia alguns
elementos balanceados na sintaxe do \LaTeX{}, retirando estruturas que exigem
abertura e fechamento de blocos. Como existe uma popularização dos formatos
Wiki, podemos trabalhar com esta linguagem e transformar o conteúdo do texto em
\LaTeX{}, apresentando um ambiente mais amigável de criação de artigos aos
usuários iniciantes.

Fornecendo a possibilidade de salvar o resultado obtido, o aluno poderá ter uma
base de estudos mais concreta, melhorando a compreensão de como os documentos
são escritos em \LaTeX{}. A exportação para documentos em PDF facilitará a
produção de documentos no formato SBC, aumentando a qualidade de trabalhos
acadêmicos.

% Bases de Estudo --------------------------------------------------------------
\section{Bases de Estudo}
\label{sec:bases-estudo}

Em 2006 no Simpósio Internacional sobre Wikis, houve o início da padronização da
linguagem de marcação de textos chamada Wiki Creole \cite{a21-junghans}. Porém,
esta padronização foi desenvolvida sem métodos formais. Com a necessidade de
formalização, foi desenvolvida uma gramática para reconhecimento da linguagem
utilizando ANTLR \cite{a4-junghans}.

Levando em consideração a popularização desta linguagem e especificação formal
completa, podemos utilizar um conjunto menor de regras desta gramática e gerar
um documento com a mesma estrutura lógica utilizando \LaTeX{}.

Um projeto semelhante usufrui de um programa capaz de transformar apresentações
criadas em \LaTeX{} com a classe Prosper em páginas Wiki, porém buscando
organizar notas de aula colaborativas \cite{p267-oneill}. Também existe um
sistema capaz de gerenciar exercícios para alunos utilizando uma extensão do
\LaTeX{}, onde um usuário pode escolher os exercícios e gerar um arquivo PDF
para impressão \cite{p161-gregorio-rodriguez}.

% Tecnologias ------------------------------------------------------------------
\section{Tecnologias}
\label{sec:tecnologias}

Primeiramente será criada uma máquina virtual para testes com o Sistema
Operacional GNU/Linux Debian. Após, deverão ser instalados um serviço de conexão
HTTP utilizando Lighttpd e possibilidade a interpretação de arquivos descritos
sobre a linguagem de programação PHP. As informações devem ser armazenadas sobre
o banco de dados PostgreSQL. O sistema será desenvolvido com a linguagem de
programação PHP utilizando Zend Framework na estrutura do servidor e Javascript
com Dojo Toolkit no cliente.

O tradutor será gerado a partir da especificação disponível sobre a Wiki Creole
com o aplicativo ANTLR e exportação alvo para linguagem PHP com a extensão
ANTLRPhpRuntime.

Para interpretação do documento resultante no formato \LaTeX{}, pacotes serão
instalados no Sistema Operacional para compilação, gerando assim os documentos
no formato PDF. A padronização do documento conforme normas da SBC dar-se-á a
partir de pacote disponibilizado pela própria instituição no formato \LaTeX{}.

% Exemplos ---------------------------------------------------------------------
\section{Exemplos}
\label{sec:exemplos}

\subsection{Entrada}

\begin{verbatim}
= Lorem ipsum =

Lorem ipsum dolor **sit amet**, consectetur adipiscing
elit. In laoreet neque sed nulla //sagittis vitae//
fermentum leo pellentesque. Ut scelerisque, felis vel
ornare ornare, mauris mauris fermentum odio, vel porta
purus lorem nec nisl.

* consectetur
*# neque
*# fermentum
* adipiscing

== Fusce mattis tincidunt sem ==

Eu sollicitudin mi tempor sit amet. Nullam mi mauris,
gravida at fringilla vulputate, viverra tempor libero.
Vivamus vitae mauris vel ante ultricies pretium ut in
metus. Nulla facilisi.
\end{verbatim}

\subsection{Saída}

\begin{verbatim}
\documentclass{article}

\usepackage[utf8]{inputenc}
\usepackage[T1]{fontenc}
\usepackage[brazil]{babel}
\usepackage{sbc-template}

\author{Enthradis Rayechir}
\title{Enthradis Rayechir's Thesis}
\address{Issnalildhatim Tatiru, Aughdbur}

\begin{document}

\maketitle{}

\section{Lorem ipsum}
\label{sec:loren-ipsum}

Lorem ipsum dolor \textbf{sit amet}, consectetur
adipiscing elit. In laoreet neque sed nulla
\textit{sagittis vitae} fermentum leo pellentesque.
Ut scelerisque, felis vel ornare ornare, mauris
mauris fermentum odio, vel porta purus lorem nec nisl.

\begin{itemize}
    \item consectetur
    \begin{enumerate}
        \item neque
        \item fermentum
    \end{enumerate}
    \item adipiscing
\end{itemize}

\subsection{usce mattis tincidunt sem}
\label{sec:usce-mattis-tincidunt-sem}

Eu sollicitudin mi tempor sit amet. Nullam mi mauris,
gravida at fringilla vulputate, viverra tempor libero.
Vivamus vitae mauris vel ante ultricies pretium ut in
metus. Nulla facilisi.

\end{document}
\end{verbatim}

% Considerações Finais ---------------------------------------------------------
\section{Considerações Finais}
\label{sec:consideracoes}

Quando é necessária a criação de um documento que possui sempre o mesmo formato,
podemos trabalhar com ferramentas que facilitam o projeto. O \LaTeX{} é uma
linguagem para marcação de textos que busca separar a lógica do documento da
formatação.

Em contrapartida, usuários iniciantes que aspiram a aprender \LaTeX{} podem
considerar a linguagem de compreensão complicada, muitas vezes pela necessidade
de balanceamento em blocos de execução.

Alguns valores como nome do autor, título da obra e endereço da instituição
podem ser omitidos do documento e salvos em outros locais, como na própria base
de dados do sistema a ser construído.

A utilização da Wiki Creole como linguagem de alto nível e que deve receber um
tradutor para a \LaTeX{} pode ser um caminho interessante na popularização de
documentos formatados utilizando determinadas estruturas padrão.

% Referências ------------------------------------------------------------------
\bibliographystyle{sbc}
\bibliography{document}

\end{document}
