\documentclass{article}

% Configurações Genéricas ------------------------------------------------------
\usepackage[utf8]{inputenc}
\usepackage[T1]{fontenc}
\usepackage[brazil]{babel}
\usepackage{sbc-template}
\usepackage{graphicx}

% Informações Pessoais ---------------------------------------------------------
\title{Especificação de Trabalho de Grau B}
\author{Wanderson Henrique Camargo Rosa\inst{1}}
\address{Linguagens de Programação 2011/1\\Centro de Ciências Exatas e
Tecnológicas\\Universidade do Vale do Rio dos Sinos ---
UNISINOS\email{wandersonwhcr@gmail.com}}

\hyphenation{antlrphpruntime postgresql}

% Documento --------------------------------------------------------------------
\begin{document}

\maketitle{}

% Introdução -------------------------------------------------------------------
\section{Introdução}
\label{sec:introducao}

Este documento tem por objetivo especificar brevemente a idéia inicial sobre o
desenvolvimento do Trabalho de Grau B desta disciplina.

Será criado um sistema \textit{web} capaz de gerenciar artigos no formato
disponível pela Sociedade Brasileira de Computação (SBC), escritos utilizando um
subconjunto da linguagem de marcação de texto Wiki Creole. O sistema deverá ser
capaz de traduzir este conteúdo para o formato LaTeX, disponibilizando-o para o
usuário. Também deve ser possível a exportação do conteúdo para arquivo no
formato PDF.

% Motivação --------------------------------------------------------------------
\section{Motivação}
\label{sec:motivacao}

Documentos técnicos e padronizados geralmente obedecem a regras específicas de
formatação. A comunidade acadêmica, em muitos casos, desconhece ferramentas que
podem facilitar a criação de documentos científicos, abstraindo a formatação dos
mesmos.

A linguagem baseada em marcação de textos \LaTeX{} pode auxiliar o autor 

% Tecnologias ------------------------------------------------------------------
\section{Tecnologias}
\label{sec:tecnologias}

Primeiramente será criada uma máquina virtual para testes com o Sistema
Operacional GNU/Linux Debian. Após, deverão ser instalados um serviço de conexão
HTTP utilizando Lighttpd e possibilidade a interpretação de arquivos descritos
sobre a linguagem de programação PHP. As informações devem ser armazenadas sobre
o banco de dados PostgreSQL. O sistema será desenvolvido com a linguagem de
programação PHP utilizando Zend Framework na estrutura do servidor e Javascript
com Dojo Toolkit no cliente.

O tradutor será gerado a partir da especificação disponível sobre a Wiki Creole
com o aplicativo ANTLR e exportação alvo para linguagem PHP com a extensão
ANTLRPhpRuntime.

Para interpretação do documento resultante no formato \LaTeX{}, pacotes serão
instalados no Sistema Operacional para compilação, gerando assim os documentos
no formato PDF. A padronização do documento conforme normas da SBC dar-se-á a
partir de pacote disponibilizado pela própria instituição no formato \LaTeX{}.

\end{document}
