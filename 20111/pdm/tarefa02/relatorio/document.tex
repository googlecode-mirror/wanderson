\documentclass{article}

% Configurações Genéricas ------------------------------------------------------
\usepackage[utf8]{inputenc}
\usepackage[T1]{fontenc}
\usepackage[brazil]{babel}
\usepackage{sbc-template}

% Informações Pessoais ---------------------------------------------------------
\title{Tarefa 02 sobre Desenvolvimento com J2ME}
\author{Wanderson Henrique Camargo Rosa\inst{1}}
\address{Programação para Dispositivos Móveis 2011/1\\Centro de Ciências Exatas
e Tecnológicas\\Universidade do Vale do Rio dos Sinos ---
UNISINOS\email{wandersonwhcr@gmail.com}}

% Documento --------------------------------------------------------------------
\begin{document}

\maketitle{}

% Introdução -------------------------------------------------------------------
\section{Introdução}
\label{sec:introducao}

Conforme tarefa da segunda semana da disciplina de Programação para Dispositivos
Móveis, existe a necessidade de criação de um aplicativo simples que deve ser
executada no simulador. Para documentação, deve-se gerar um relatório de
atividade relatando os passos realizados.

O presente documento é resultado final desta tarefa e busca relatar sobre os
serviços executados durante o desenvolvimento do \emph{software}.

% Instalação do Ambiente -------------------------------------------------------
\section{Instalação do Ambiente}
\label{sec:instalacao}

O ambiente de desenvolvimento escolhido foi Eclipse, plataforma que já utilizo
em outras áreas de desenvolvimento. Visitando a página de \emph{download}, a
primeira idéia que sempre tive foi que o JavaEE seria o ambiente para criação de
aplicativos para dispositivos móveis, o que foi brevemente foi retirado quando
li alguns textos sobre.

Resolvi copiar a versão para desenvolvimento em Java. Logo após, procurei
informações sobre como desenvolver aplicativos J2ME no ambiente. Encontrei o
\emph{plugin} MTJ, sucessor do EclipseME. O gerenciamento de pacotes do projeto
MTJ traz alguns problemas quanto a atualização entre versões, o que é impossível
porque o endereço URL é fixo para cada versão. Após instalar a última versão,
tentei gerar um novo projeto, sem sucesso.

Para criação deste é necessário um dispositivo para emulação que não estava
instalado. Eu particularmente achava que eles já teriam sido instalados junto
com o \emph{plugin}. Após compreender o problema, resolvi ler a documentação
disponível na tarefa e encontrei a WDK. Para instalação, a WDK exige o diretório
onde a máquina virtual Java está instalada. O Ubuntu salva este diretório em um
local diferenciado, possivelmente visando velocidade.

O Eclipse necessita de configurações para encontrar o caminho da WDK

% Sobre o Aplicativo -----------------------------------------------------------
\section{Sobre o Aplicativo}
\label{sec:sobre}

Conforme primeira aula presencial da disciplina, o J2ME receberá uma atenção
menor porque está sendo menos usado hoje em dia. Outras tecnologias serão
inseridas no conteúdo com o passar das aulas. Porém, nesta semana, devemos criar
uma aplicação simples.

\subsection{Apresentador de Slides}

A primeira idéia de desenvolvimento 

\end{document}