\documentclass{article}

\usepackage[utf8]{inputenc}
\usepackage[T1]{fontenc}
\usepackage[brazil]{babel}
\usepackage{sbc-template}

\title{Proposta de Implementação de Tradutor Java para Linguagem Intermediária
em XML}
\author{Roberto Raguze Flores\inst{1}\\Wanderson Henrique Camargo Rosa\inst{1}}
\address{Centro de Ciências Exatas e Tecnológicas\\Universidade do Vale do
Rio dos Sinos (UNISINOS)\email{\{roberto.raguze,wandersonwhcr\}@gmail.com}}

\begin{document}

\maketitle{}

\section{Motivação}

Conforme solicitação do Prof Gustavo Lermen, através da disciplina de Tradutores
do curso de Ciência da Computação desta instituição, segundo semestre de 2010,
é necessária a criação de um tradutor qualquer que implemente estruturas de 
desenvolvimento vistas em aula. Também é requerido um artigo que descreva este
tradutor. Portanto, este documento descreve de forma sucinta o trabalho
supracitado.

\section{Introdução}

Atualmente, o desenvolvimento de programas de forma coletiva pode ser
administrado através de \emph{softwares} para versionamento de código fonte, que
visam registrar cada modificação do programador, adicionando informações como
autor ou comentários sobre aqueles novos dados inseridos.

\section{Linguagem de Origem}

\section{Linguagem Alvo}

\section{Exemplos de Entrada e Saída}

\section{Metodologia Esperada}

\end{document}
