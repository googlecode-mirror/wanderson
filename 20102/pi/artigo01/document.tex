\documentclass{article}

\usepackage[utf8]{inputenc}
\usepackage[T1]{fontenc}
\usepackage[brazil]{babel}
\usepackage{sbc-template}

\title{Protótipo de \emph{Interface} Humano Computador para Supermercados
Utilizando Técnicas de Lojas Virtuais e Computação Ubíqua}
\author{Wanderson Henrique Camargo Rosa \and Roberto Raguze Flores}
\address{Centro de Ciências Exatas e Tecnológicas\\Universidade do Vale do Rio
dos Sinos (UNISINOS)\email{\{wandersonwhcr \and roberto.raguze\}@gmail.com}}

\begin{document}

\maketitle{}

\section{Introdução}

% Contextualização Geral

A utilização de ambientes virtuais para compras através de lojas eletrônicas
está cada vez mais comum nos dias de hoje. Através deste método de venda é
possível atingir mais facilmente o público alvo com promoções, dicas de produtos
semelhantes e comparação de preços.

% Negativos de Lojas Virtuais

Porém estes ambientes não fornecem ao cliente a principal característica de
venda de um produto: o contato direto. O ato do contato físico com produto pode
ser a ação chave para a compra do mesmo ou escolha de outros itens semelhantes,
algo que somente um comércio tradicional pode nos fornecer.

Se por um lado a informatização nos traz a facilidade de comunicar as pessoas
sobre o produto em questão, a venda tradicional pode nos dar uma melhor
segurança durante a compra. Porém, durante a compra comum, clientes enfrentam a
falta de informações de um determinado produto, recorrendo à pesquisa de dados
adicionais em lojas virtuais \cite{vonreischach2009}.

% União das Idéias

Dado que ambos os lados possuem características interessantes, podemos
desenvolver uma \emph{interface} para uma loja comum, trazendo as facilidades
virtuais para o mundo real, através de ambientes de realidade aumentada.

\section{Interação Humano Computador}

% Realidade

Em um estabelecimento real, um produto fica disponível ao consumidor, podendo
este interagir com o objeto. Informações adicionais podem ser fornecidas por
funcionários da loja, contudo, nem sempre a pessoa responsável pela venda
conhece completamente todos os produtos da loja ou pode fornecer dados técnicos
especializados, muito menos sugestões de todos os clientes que já compraram o
objeto.

% Realidade Aumentada

Por meio de dispositivos móveis, o cliente poderá acessar informações
pertinentes a qualquer produto da loja, pesquisar sobre produtos disponíveis e
em promoção, bem como controlar sua compra atual.

As informações serão acessadas em tempo real e os dados do cliente podem ser
salvos, visando criar ambientes do sistema especializados para cada pessoa,
estudando os produtos já comprados por ela e sugerindo novos. Com o cadastro
também é possível que o cliente efetue um pagamento \emph{online} dos objetos,
evitando filas em caixas de pagamento.

\subsection{Acesso aos Dados}

% Como é a captura

O cliente, ao entrar na loja, terá duas opções de acesso ao sistema: utilizando
um leitor óptico fornecido ou instalando um aplicativo em seu próprio celular,
desde que este possua requisitos como acesso \emph{wi-fi}, câmera digital e
navegador \emph{web}. Estas são as características dos celulares atuais,
chamados \emph{smartphones}.

% Sistema Deverá Fazer

O sistema estará disponível utilizando uma rede sem fios interna ao recinto. Ele
deve armazenar dados sobre todos os produtos disponíveis na loja e controlar as
compras dos clientes. Portanto, cada pessoa que está em compras no
estabelecimento deverá efetuar autenticação no sistema, utilizando o dispositivo
móvel.

% Utilização de QRCode

Para acesso facilitado às informações do produto no dispositivo, cada objeto
deverá receber um código de barra de duas dimensões \cite{alapetite2010}. Estes
códigos contém um endereço no sistema interno para acesso à descrição, onde o
cliente poderá inclusive selecionar a quantidade desejável de objetos e
adicioná-los ao carrinho de compras virtual e real, conforme seja necessário.

% Ações Possíveis

Utilizando esta característica, o cliente poderá acessar informações mais
detalhadas do produto, como dados técnicos ou valores energéticos, comparar
preços e melhores dias para compras, bem como fornecer sugestões do produto para
terceiros \cite{canny2006} ou acessar futuras promoções da loja.

% Conclusão do Acesso aos Dados

Assim, o rápido acesso às informações de produto poderão ser incluídas em
ambientes reais, trazendo os meios virtuais para a realidade, extendendo-a. O
usuário terá todos os dados disponíveis do produto que está comprando e poderá
vê-lo, unindo o melhor dos dois ambientes.

\subsection{Idealização}

Com esta análise de informações, há uma proposta de criação de uma
\emph{interface} para supermercados que utilize técnicas de lojas virtuais,
tornando a compra mais fácil e controlada, através de dispositivos móveis.

Temos então como público alvo todas as pessoas que utilizam o ambiente do
supermercado e que já estão acostumadas aos padrões comuns. O sistema deve ser
capaz de facilitar as compras, dando comodidade e segurança ao cliente.

Não podemos esquecer que nem todas as pessoas estão acostumadas com tecnologias
diferenciadas. Portanto, quanto mais a \emph{interface} estiver adaptável ao
usuário, melhor será o retorno e satisfação do cliente.

\section{Protótipo}

\section{Conclusão}

\bibliographystyle{sbc}
\bibliography{sbc-template}

\end{document}