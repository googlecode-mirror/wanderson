\documentclass{article}

% Pacotes Utilizados -----------------------------------------------------------
\usepackage[utf8]{inputenc}
\usepackage[T1]{fontenc}
\usepackage[brazil]{babel}
\usepackage{sbc-template}

% Dados Pessoais ---------------------------------------------------------------
\title{Proposta de Análise de \textit{Sites} de Prefeituras Municipais}
\author{Wanderson Camargo\inst{1}\and{}Roberto Raguze\inst{1}\and{}Daniel
Perazzoni\inst{1}}
\address{Centro de Ciências Exatas e Tecnológicas\\
Universidade do Vale do Rio dos Sinos --- UNISINOS
\email{\{wandersonwhcr\and{}roberto.raguze\and{}dperazzoni\}@gmail.com}}

\begin{document}

\maketitle{}

% Introdução -------------------------------------------------------------------
\section{Introdução}

Atualmente, a Internet traz aos seus usuários uma maior velocidade na troca de
informações importantes. Notícias e serviços podem ser acessados rapidamente
através de um \textit{site} ou portal, onde estes dados são exibidos
publicamente ou recebidos através de serviços como \textit{feeders} ou
\textit{newsletters}.

O setor público pode utilizar desta facilidade para otimizar alguns serviços
disponíveis de atendimento. Informações como telefone para contato, endereços de
Secretarias ou formulário para solicitação de serviço poderão estar disponíveis
em portal na Internet para o cidadão. Notícias da Prefeitura, atuais serviços
prestados ou encontros de representantes com a população tornam o trabalho deste
órgão municipal mais transparente.

Uma \textit{interface} neste caso, deve fornecer uma usabilidade relativamente
acessível para que qualquer pessoa deste município consiga executar tarefas
simples, como consulta de telefones para contato.

Esta proposta de trabalho visa analisar portais de prefeituras municipais da
região metropolitana de Porto Alegre e da região do Vale do Rio dos Sinos. Elas
serão estudadas igualmente, buscando pontos específicos de comunicação com o
cidadão e verificando se o usuário possui uma facilidade ou não de utilização da
sua \textit{interface}.

% Descrição --------------------------------------------------------------------
\section{Descrição}

A análise será feita utilizando como base 3 \textit{sites} de prefeituras
municipais da região metropolitana de Porto Alegre e da região do Vale do Rio
dos Sinos, que serão analisados com foco na qualidade e adequação dos problemas
do usuário final, cidadão residente do município alvo.

\begin{itemize}
  \item Prefeitura Municipal de Porto Alegre
  \item Prefeitura Municipal de Gravataí
  \item Prefeitura Municipal de São Leopoldo
\end{itemize}

\subsection{Assuntos}

Primeiramente, serão tratadas as buscas de informações para contato com os
órgãos da prefeitura, como telefones diretos para serviços ou formulários para
ocorrência de problemas. Num segundo momento, serão discutidas como as notícias
da prefeitura são distribuídas, como são atualizadas e se o conteúdo é
consistente. Após serão analisadas formas de acesso a dados, como leis 
municipais e gastos com recursos públicos atualizados em tempo real, sistema
atualmente chamado \textit{Portal de Transparência}.

% Técnicas Utilizadas ----------------------------------------------------------
\section{Técnicas Utilizadas}

Como estes serviços já estão finalizados e disponíveis, será aplicada uma
avaliação somativa, levando em consideração a facilidade do uso de seu conteúdo.
O método trabalhado possui características de inspeção de usabilidade, que será
aplicado em 5 voluntários residentes de cada cidade.

Portanto, como técnica utillizada, teremos testes de usabilidades em usuários
finais com preenchimento de questionário previamente formatado. Os três assuntos
serão trabalhados com 6 questões cada, com uma linguagem não técnica, visando
manter o usuário a vontade durante o tempo de respostas.

% Casos Analisados -------------------------------------------------------------
\subsection{Casos Analisados}

\begin{itemize}
  \item Informações
  \begin{itemize}
    \item Disponibilidade do \textit{site} em mecanismos de busca;
    \item Apresentação da página inicial;
    \item Mecanismo interno de busca de conteúdo;
    \item Descrição de todas as seções do \textit{site};
    \item Telefone principal para contato; e
    \item Formulário de contato para sugestões ou dúvidas.
  \end{itemize}
  \item Notícias
  \begin{itemize}
    \item Disponibilidade;
    \item Fácil compreensão;
    \item Atualização periódica;
    \item Notícias por região;
    \item Pesquisa de notícias por conteúdo; e
    \item Impressão das notícias.
  \end{itemize}
  \item Serviços
  \begin{itemize}
    \item Acesso às leis municipais;
    \item Serviços de atendimento ao cidadão;
    \item Acesso ao PROCON;
    \item Acompanhamento de pedido de documentos e protocolo;
    \item Telefones úteis para postos de saúde ou escolas da região; e
    \item Portal de transparência com atualização em tempo real.
  \end{itemize}
\end{itemize}

% Documentação Final -----------------------------------------------------------
\section{Documentação Final}

Como documentos desta pesquisa serão gerados um artigo com relatório da análise
dos dados fornecidos pelos usuários e gráficos demonstrando as facilidades e os
problemas encontrados. Também serão anexados os questionários preenchidos.

Como finalização, serão apresentadas melhorias para estes portais caso hajam
falhas de usabilidade do ponto de vista dos usuários.

\end{document}