\documentclass{article}

% Pacotes Utilizados -----------------------------------------------------------
\usepackage[utf8]{inputenc}
\usepackage[T1]{fontenc}
\usepackage[brazil]{babel}
\usepackage{sbc-template}

% Dados Pessoais ---------------------------------------------------------------
\title{Proposta de Análise de Interface}
\author{Wanderson Henrique Camargo Rosa\inst{1}}
\address{Universidade do Vale do Rio dos Sinos -- UNISINOS}

\begin{document}

\maketitle{}

% Introdução -------------------------------------------------------------------
\section{Introdução}

Atualmente, a Internet traz aos seus usuários uma maior velocidade na troca de
informações importantes. Notícias e serviços podem ser acessados rapidamente
através de um \textit{site} ou portal, onde estes dados são exibidos
publicamente ou recebidos através de serviços como \textit{feeders} ou
\textit{newsletters}.

O setor público pode utilizar desta facilidade para otimizar alguns serviços
disponíveis de atendimento ao público. Informações como telefone para contato,
endereços de Secretarias ou formulário para solicitação de serviço poderão estar
disponíveis em portal na Internet para o cidadão. Notícias da Prefeitura, atuais
serviços prestados ou encontros com a população de seus representantes tornam o
trabalho deste órgão municipal mais transparente.

Uma \textit{interface} neste caso, deve fornecer uma usabilidade relativamente
acessível para que qualquer pessoa deste município consiga executar tarefas
simples, como consulta de telefone para contato.

Esta proposta de trabalho visa analisar portais de prefeituras municipais da
região metropolitana de Porto Alegre e da região do Vale do Rio dos Sinos. Elas
serão estudadas igualmente, buscando pontos específicos de comunicação com o
cidadão e verificando se o usuário possui uma facilidade ou não de utilização da
sua \textit{interface}.

% Descrição --------------------------------------------------------------------
\section{Descrição}

A análise será feita utilizando como base 3 \textit{sites} de prefeituras
municipais da região metropolitana de Porto Alegre e da região do Vale do Rio
dos Sinos, que serão analisados com foco na qualidade e adequação dos problemas
do usuário final, cidadão residente do município alvo.

\begin{itemize}
  \item Prefeitura Municipal de Canoas
  \item Prefeitura Municipal de Gravataí
  \item Prefeitura Municipal de São Leopoldo
\end{itemize}

\subsection{Assuntos}

Primeiramente, serão tratadas as buscas de informações para contato com os
órgãos da prefeitura, como telefones diretos para serviços ou formulários para
ocorrência de problemas. No segundo momento, serão discutidas como as notícias
da prefeitura são distribuídas, como são atualizadas e se o conteúdo é
consistente. Após serão analisadas formas de acesso a dados como leis municipais
e gastos atualizados em tempo real, chamado \textit{Portal de Transparência}.

% Técnicas Utilizadas ----------------------------------------------------------
\section{Técnicas Utilizadas}

Como estes serviços já estão terminados e disponíveis, será aplicada uma
avaliação somativa, levando em consideração a facilidade de uso de seu conteúdo.
O método trabalhado possui características de inspeção de usabilidade, que será
aplicado em 5 voluntários residentes de cada cidade.

Portanto, como técnica utillizada, teremos testes de usabilidades em usuários
finais com preenchimento de questionário previamente formatado. Os primeiros
três assuntos serão trabalhados com 10 questões cada.

% Casos Analisados -------------------------------------------------------------
\section{Casos Analisados}

\begin{itemize}
  \item Telefone para contato;
  \item Telefone de atendimentos médicos;
  \item Formulário de contato ou email;
  \item Notícias atualizadas;
  \item Pesquisa de notícias ocorridas há 2 semanas;
  \item Acesso ao PROCON;
  \item Acesso às leis municipais;
  \item Portal de Transparência e dados em tempo real;
  \item Acesso a portadores com deficiência visual;
\end{itemize}

% Documentação Final -----------------------------------------------------------
\section{Documentação Final}

Como documentos desta pesquisa serão gerados um artigo com relatório da análise
dos dados fornecidos pelos usuários e gráficos demonstrando as facilidades e os
problemas encontrados. Também serão anexados os questionários preenchidos.

\end{document}