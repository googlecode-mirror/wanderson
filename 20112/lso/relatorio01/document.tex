\documentclass{article}

\usepackage[utf8]{inputenc}
\usepackage[T1]{fontenc}
\usepackage[brazil]{babel}
\usepackage{sbc-template}

\title{Relatório de TGA sobre Shellscript}
\author{Wanderson Henrique Camargo Rosa\inst{1}}
\address{Laboratório de Sistemas Operacionais --- 2011/2\\Centro de Ciências
Exatas e Tecnológicas\\Universidade do Vale do Rio dos Sinos --- UNISINOS}

\begin{document}

\maketitle{}

\section{Introdução}

Este documento visa apresentar anotações e criar um relatório parcial das
atividades propostas pela disciplina de Laboratório de Sistemas Operacionais,
semestre de 2011/2, sobre programação Shellscript.

Na proposta do trabalho, foi apresentado um código-fonte de um programa descrito
em Shellscript com alguns problemas de sintaxe e codificação, sendo que existe a
necessidade de organização de código, refatoração e documentação em blocos de
código, visando estudos sobre o algoritmo implementado.

\section{Análise Geral}

O \textit{script} atual busca criar um novo método para execução de
\textit{downloads} sobre arquivos informados em linha de comando. Recebendo como
parâmetro opcional \texttt{-t}, podemos esperar a cópia do arquivo remoto pela
quantidade de segundos informada. Após este parâmetro, podemos fornecer uma
lista de arquivos para download, processados pelo \textit{script}.

Utilizando um laço de repetição, o bloco principal do programa desloca os
parâmetros para esquerda, afim de fornecê-lo ao programa real que efetua o
\textit{download} dos arquivos, neste caso, o \texttt{wget}. Este, é
inicializado com o parâmetro \texttt{-c} que tenta continuar um
\textit{download} pausado anteriormente.

Ainda neste \textit{script}, existem duas funções implementadas, onde a primeira
utiliza o programa para executar o \textit{download} sem tempo de término,
chamado de \textit{timeout}. A segunda trabalha com \textit{timeout} simulado,
utilizando uma espera de execução no sistema operacional do processo que é
enviado ao \textit{background}. Quando este \textit{timeout} é finalizado, o
programa de \textit{download} é finalizado através de um sinal \texttt{KILL}.

Estas funções são chamadas para cada arquivo informado na linha de comando. A
saída de erro do \texttt{KILL} é redirecionada para o escritor especial que
ignora as informações enviadas \texttt{/dev/null}. Já o programa que executa o
\textit{download} apresenta normalmente a sua saída. Ao final, são apresentados
os resultados na saída padrão, efetuando a soma da quantidade de
\textit{downloads} executados e quantos foram finalizados com sucesso.

\section{Tópicos}

Houve a necessidade de criação de um arquivo com o nome de
\texttt{downloader.sh} para armazenar o código-fonte informado pela disciplina.
Para que o arquivo receba credenciais para execução, houve a necessidade de
execução do comando \texttt{chmod +x downloader.sh} ou chamar diretamente o
interpretador de comandos com o comando \texttt{bash downloader.sh}.

O programa atual não possui a linha de \textit{hashbang} completo, sendo
necessária sua modificação para \texttt{\#!/bin/bash}.

As variáveis devem ser inicializadas com o sinal de atribuição como sufixo e
adjacente, sem espaços entre ambos os \textit{tokens}, caso contrário a
atribuição desejada não irá funcionar. Também existe o caso em que variáveis
recebem valores e o sinal de atribuição está sendo separado por um espaço, não
categorizando somente as inicializações.

O código de término de execução com falhas deve ser diferente de zero;
necessário porque o sistema operacional espera um inteiro com valor 0 para que o
processo informe sucesso em sua execução. Já o código de sucesso também deve ser
alterado, buscando apresentar um valor conciso ao sistema operacional sobre o
estado do processamento.

O nome da variável de sucesso está sendo inicializado com identificador
diferente de todos os outros pontos do código-fonte, gerando uma inconsistência
na execução correta, onde outros pontos do programa irão utilizar variáveis não
inicializadas. Isto também está acontecendo na apresentação de resultado com
falha, onde o conteúdo da variável não é expandido.

A função \texttt{print\_info} possui retorno que não está sendo acessado
corretamente através da expansão de variável, não acessando o valor de sucesso
armazenado no conteúdo.

Necessária implementação da divisão entre os dois blocos condicionais do
\texttt{case}, buscando não executar os dois comandos em cascata e saindo do
bloco. Neste caso, precisamos que o \texttt{case} não execute os próximos
comandos, portanto é preciso adicionar dois pontos-e-vírgula \texttt{(;;)}
para escapar do bloco condicional.

Quando existe a verificação da quantidade de parâmetros que foram informados ao
\textit{script}, existe a necessidade de escapar o sinal de menor-que
\texttt{(<)} porque o interpretador de comandos espera que o segundo parâmetro
seja um arquivo para redirecionamento da entrada padrão.

O número para acesso dos parâmetros dentro da função \texttt{print\_info} está
errado, não informando corretamente o número de \textit{downloads} executados
com sucesso, muito menos executando o somatório de \textit{downloads}. Aplicada
documentação mais completa ao \textit{script}, com descrição de funcionalidades
executadas.

A primeira função efetua o \textit{download} dos arquivos sem \textit{timeout};
logo o sistema operacional espera até que a cópia do arquivo remoto seja
finalizada. Podemos utilizar o parâmetro opcional \texttt{-t} na primeira
função, porém ela não é manipulada dentro de seu escopo. Para corrigir o seu
problema de execução, somente foi adicionado o parâmetro de \textit{timeout} na
chamada para execução, tendo em vista que outros pontos do sistema podem
utilizar a mesma função.

Finalizando, o último \texttt{shift} não aparenta ser necessário porque não
existe manipulação de parâmetros após a sua chamada.

\end{document}
