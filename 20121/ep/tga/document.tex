\documentclass{article}

% Pacotes Utilizados -----------------------------------------------------------
\usepackage[brazil]{babel}
\usepackage[utf8]{inputenc}
\usepackage[T1]{fontenc}
\usepackage{sbc-template}

% Informações Pessoais ---------------------------------------------------------
\title{Resumo}
\author{Wanderson Henrique Camargo Rosa\inst{1}}
\address{Universidade do Vale do Rio dos Sinos --- UNISINOS}

% Início do Documento ----------------------------------------------------------
\begin{document}

% Título -----------------------------------------------------------------------
\maketitle{}

% Introdução -------------------------------------------------------------------
\section{Introdução}
\label{sec:intro}

Este documento visa apresentar uma etapa inicial para estado da arte, visando
apresentar informações sobre a proposta para Trabalho de Conclusão de Curso do
presente autor. Serão resumidos alguns trabalhos encontrados da área.

% Ensinamentos -----------------------------------------------------------------
\section{Ensinamentos}
\label{sec:ensin}

O artigo apresentado por \cite{gray2003} discute sobre experiências e lições
aprendidas durante aulas sobre \LaTeX{} ministradas em 2001 na Universidade
Estadual da Pensilvânia, para 15 alunos e utilizando carga horária referente a 1
crédito. Há a discussão sobre quais materiais foram utilizados para execução da
aula, recursos utilizados no preparo do material e quais as tarefas apresentadas
aos alunos. Além disso, são apresentadas informações sobre as dificuldades
encontradas pelos alunos, com base em retornos provenientes dos alunos,
relacionando o artigo à proposta de TCC.

\subsection{Detalhe do Problema}

Os autores são responsáveis na supervisão de documentos técnicos de estudantes
universitários e pós-graduação da Universidade, possuindo 19 anos de experiência
com \LaTeX{}. No escopo aplicado, consideram-se documentos técnicos aqueles que
possuem muitas equações e figuras disponíveis. Muitos dos alunos estavam
utilizando documentos no formato Microsoft Word\texttrademark{}; não discordam
que a ferramenta pode produzir um documento com boa qualidade utilizando todas
as ferramentas disponíveis pelo aplicativo, porém afirmam que muitos dos alunos
não sabem que tais ferramentas estão disponíveis, como estilizações e editores
de equação.

Por reconhecerem que \LaTeX{} é uma excelente ferramenta para construção de
documentos mais longos, como dissertações, juntamente com a curiosidade dos
alunos que visualizam suas apostilas de alta qualidade, criaram uma disciplina
optativa que ensina como utilizar \LaTeX{}, entitulada \textit{Technical
Documents with \LaTeX{}}, no final do primeiro semestre de 2001. Estas aulas
possuiram carga horária de 75 minutos, uma vez por semana, com 7 encontros. As
avaliações foram baseadas em tarefas extraclasse com 75\% da nota final e
participações em sala de aula com 25\%.

A estrutura física contava com \textit{laptops} individuais com \LaTeX{} e
acesso à Internet. A base teórica foi desenvolvida sobre os livros [KOPKA AND
DALY 1999] e a parte matemática com [GRATZER 2000]. Para auxiliar os alunos, foi
disponibilizado um site com documentos de exemplo e assuntos apresentados em
aula.

O primeiro encontro baseou-se na apresentação da linguagem tipográfica com base
em tutoriais, documentos de exemplo e tarefas extraclasses. Os requisitos para
utilização exigidos dos alunos eram gerenciar arquivos em computadores,
instalação e execução de aplicativos e editores de texto simples.

As tarefas extraclasse apresentadas foram criadas buscando poderem ser
resolvidas em tempo não superior a 3 horas, dando oportunidades de contatar os
ministrantes caso o tempo de resolução utilizado estiver sendo ultrapassado.

Com dificuldades, alunos informaram que estavam levando 9 horas para executar as
tarefas e que ainda não haviam terminado. Porém, eles não buscaram procurar os
documentos disponíveis no site e nem auxílio dos ministrantes.

Vendo estas dificuldades, no próximo encontro buscaram coletar mais informações
sobre os problemas encontrados e criaram um ponto de discussão sobre
características mais simples e separadas da linguagem, como estruturação geral,
ambientes e pacotes. Ao final deste encontro, buscaram unir todas as informações
criando assim um documento final. A partir do terceiro encontro, o foco de
utilização foi direcionado para estruturas matemáticas e equações. No quinto
encontro trabalhou-se a inclusão de gráficos em \LaTeX{} e o sexto buscou-se
incluir customizações em comandos e bibliografias. No último encontro, foi
apresentado o pacote da ferramenta para construção de Teses no formato utilizado
pela Universidade.

Durante as aulas, a utilização livre do acesso à Internet foi notado como um
problema para o desvio de atenção dos alunos. Após solicitarem aos alunos que
não utilizassem tais elementos não pertencentes à aula, houve um maior respeito
com os ministrantes.

\subsection{Retornos}

Ao final do curso, são apresentados questionários respondidos anonimamente pelos
alunos, coletados com informações sobre as aulas. Nestes, foram encontrados
comentários sobre a utilização, enfatizando a utilização do site, a apresentação
da ferramenta utilizando exemplos, curso assumindo que os alunos não possuem
conhecimentos da ferramenta, alternativa aos editores de texto mais populares e
que ficaram felizes em aprender algo que pode ser útil em suas carreiras.

Pontos negativos também foram apresentados, como não instrução de como utilizar
pacotes específicos para \LaTeX{}, reclamações de que o curso possui muitas
tarefas para único crédito e que as tarefas extraclasse levam muito tempo. Com
base nestas informações, sugestões foram apresentas, visando aumentar a carga
horária para 12 encontros ou um semestre completo e diminuição de tarefas
extraclasse que deveria ser disponibilizada mais gradualmente.

Com base nestas informações, concluiu-se que deveriam aumentar a carga horária
do curso, tendo em vista que as aulas foram ministradas de uma forma muito
rápida. A quantidade de alunos deveria ser diminuída e possuir mais encontros
semanais, buscando assim melhorar o atendimento individual. Além disso, deveriam
auxiliar mais os alunos que possuem dificuldades na utilização do \LaTeX{} em
seus próprios computadores.

Concluindo as aplicações, apresentaram que o curso ministrado deveria ser
considerado como extra curricular. Consultas posteriores informam que cerca de
metade dos alunos que assistiram ao curso ainda continuam a utilizar a
ferramenta, caracterizando assim um entusiasmo por parte deles.

% Bibliografia -----------------------------------------------------------------
\bibliographystyle{sbc}
\bibliography{document}

\end{document}

