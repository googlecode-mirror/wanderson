\documentclass{article}

% Pacotes Utilizados -----------------------------------------------------------
\usepackage[brazil]{babel}
\usepackage[utf8]{inputenc}
\usepackage[T1]{fontenc}
\usepackage{sbc-template}

% Informações Pessoais ---------------------------------------------------------
\title{Facilitador para Escrita de Artigos Científicos: Estado da Arte}
\author{Wanderson Henrique Camargo Rosa\inst{1}}
\address{
Elaboração de Projetos --- 2012/1\\
Centro de Ciências Exatas e Tecnológicas\\
Universidade do Vale do Rio dos Sinos --- UNISINOS
\email{wandersonwhcr@gmail.com}
}

% Início do Documento ----------------------------------------------------------
\begin{document}

% Título -----------------------------------------------------------------------
\maketitle{}

% Introdução -------------------------------------------------------------------
\section{Introdução}
\label{sec:intro}

Este documento visa apresentar uma etapa inicial para estado da arte, visando
apresentar informações sobre a proposta para Trabalho de Conclusão de Curso do
autor. Serão resumidos alguns trabalhos encontrados da área de estudos sobre a
linguagem de marcação \LaTeX{} e possíveis aplicações da linguagem Wiki Creole
como fonte para tradução, criando assim artigos científicos de uma forma mais
simples.

A Seção \ref{sec:falha} dialoga sobre a curva de aprendizado inicial que a
linguagem de marcação \LaTeX{} possui e as mensagens de erro obscuras
apresentadas quando um documento não está bem escrito, referenciadas na Seção
\ref{sec:erros}. Métodos de ensino e possíveis problemas encontrados durante um
curso criado para aprendizado \LaTeX{} é descrito na Seção \ref{sec:ensin}. Um
conversor de formato \LaTeX{} para Wiki Creole foi criado para que alunos
comentem em páginas utilizando o formato alvo as anotações de aula que estão
sendo anotadas naquele momento é descrito na Seção \ref{sec:conve}. Padronizando
a Wiki Creole, temos sua proposta de definição dialogada na Seção
\ref{sec:padro}. Ao final, apresenta-se uma conclusão sobre os artigos aqui
apresentados na Seção \ref{sec:concl}.

% Falhas -----------------------------------------------------------------------
\section{Falhas}
\label{sec:falha}

Segundo \cite{salzberg2005}, o Microsoft Word\texttrademark{} pode ser utilizado
na produção de documentos numa variedade de necessidades, porém possui algumas
limitações. Nestas ocasiões, o \TeX{} pode ser uma alternativa. Ele é um padrão
de formato para algumas disciplinas acadêmicas, particularmente aquelas que
utilizam matemática.

Escrever um documento utilizando \TeX{} requer pensar sobre o conteúdo e
formatação de forma separada, liberando o autor para somente escrever. \LaTeX{}
é um conjunto de macros e programas que fazem a utilização do \TeX{} mais
simples, gerando facilidades para documentos mais genéricos, como artigos e
livros. Ainda segundo \cite{salzberg2005}, \TeX{} possui uma curva de
aprendizado íngreme, facilitando a inicialização do uso se o usuário possui
experiência em outros tipos de marcação de texto.

% Erros ------------------------------------------------------------------------
\section{Erros}
\label{sec:erros}

Conforme \cite{ha2007}, um erro de sistema ocorre quando o \textit{software} não
pode completar uma ação requisitada, como um resultado de algum problema na
entrada, configuração ou ambiente. Melhores informações sobre erros encontrados
possibilita ao usuário entender e corrigir o problema, porém esta qualidade é
inversamente proporcional à criação de \textit{softwares} mais complexos.

Para aplicar uma tentativa de solução, o autor apresenta um sistema que melhora
A apresentação de erros, classificando os comportamentos do aplicativo, chamado
Clarify. Este usa um monitoramento invasivo mínimo para gerar um perfil de
comportamento, que pode ser anexado ao histórico de execução do programa
\cite{ha2007}.

O sistema pode ser analisado em duas partes: um monitor em tempo de execução e
um classificador que utiliza aprendizado de máquina e que interpreta a saída da
execução. Clarify monitora o programa executando técnicas invasivas, como
leitura de memória do programa ou contabilização de chamadas na pilha do
processo.

Conforme relatos do autor, o sistema foi executado sob mensagens ambíguas em
muitos aplicativos, inclusive \LaTeX{}, com custo de desempenho de menos de 1\%
o protótipo conseguiu melhor identificar as mensagens de erro em pelo menos 85\%
dos comportamentos que resultaram como ambíguas.

Utilizando o Clarify com \LaTeX{}, o autor descreve 26 casos de erro duplicados.
Por exemplo quando uma tabela, vetor ou vetor de equações possui maior
quantidade de separadores de células do que a quantidade definida de colunas, o
\LaTeX{} apresenta um erro de alinhamento, onde o caractere de final de linha
está numa posição inválida. Este mesmo erro é apresentado quando o usuário
esquece de incluir um símbolo de quebra de linha ao final das colunas.

Na descrição do sistema de tipografia, ele afirma que a exibição de erros é
muito conhecida por ser obscura. Ele também apresenta um aplicativo chamado
Rubber que filtra a saída padrão do \LaTeX{} para torná-la mais compreensível ao
usuário. Todavia, muitas mensagens de erro são genéricas e podem ter muitas
causas, dificultando ao usuário a compreensão dos acontecimentos e aplicar uma
correção \cite{ha2007}.

% Ensinamentos -----------------------------------------------------------------
\section{Ensinamentos}
\label{sec:ensin}

O artigo apresentado por \cite{gray2003} discute sobre experiências e lições
aprendidas durante aulas sobre \LaTeX{} ministradas em 2001 na Universidade
Estadual da Pensilvânia, para 15 alunos e utilizando carga horária referente a 1
crédito. Há a discussão sobre quais materiais foram utilizados para execução da
aula, recursos utilizados no preparo do material e quais as tarefas apresentadas
aos alunos. Além disso, são apresentadas informações sobre as dificuldades
encontradas pelos alunos, com base em retornos provenientes dos alunos,
relacionando o artigo à proposta de TCC.

\subsection{Detalhe do Problema}

Os autores são responsáveis na supervisão de documentos técnicos de estudantes
universitários e pós-graduação da Universidade, possuindo 19 anos de experiência
com \LaTeX{}. No escopo aplicado, consideram-se documentos técnicos aqueles que
possuem muitas equações e figuras disponíveis. Muitos dos alunos estavam
utilizando documentos no formato Microsoft Word\texttrademark{}; não discordam
que a ferramenta pode produzir um documento com boa qualidade utilizando todas
as ferramentas disponíveis pelo aplicativo, porém afirmam que muitos dos alunos
não sabem que tais ferramentas estão disponíveis, como estilizações e editores
de equação.

Por reconhecerem que \LaTeX{} é uma excelente ferramenta para construção de
documentos mais longos, como dissertações, juntamente com a curiosidade dos
alunos que visualizam suas apostilas de alta qualidade, criaram uma disciplina
optativa que ensina como utilizar \LaTeX{}, entitulada \textit{Technical
Documents with \LaTeX{}}, no final do primeiro semestre de 2001. Estas aulas
possuiram carga horária de 75 minutos, uma vez por semana, com 7 encontros. As
avaliações foram baseadas em tarefas extraclasse com 75\% da nota final e
participações em sala de aula com 25\%.

A estrutura física contava com \textit{laptops} individuais com \LaTeX{} e
acesso à Internet. A base teórica foi desenvolvida sobre os livros [KOPKA AND
DALY 1999] e a parte matemática com [GRATZER 2000]. Para auxiliar os alunos, foi
disponibilizado um site com documentos de exemplo e assuntos apresentados em
aula.

O primeiro encontro baseou-se na apresentação da linguagem tipográfica com base
em tutoriais, documentos de exemplo e tarefas extraclasses. Os requisitos para
utilização exigidos dos alunos eram gerenciar arquivos em computadores,
instalação e execução de aplicativos e editores de texto simples.

As tarefas extraclasse apresentadas foram criadas buscando poderem ser
resolvidas em tempo não superior a 3 horas, dando oportunidades de contatar os
ministrantes caso o tempo de resolução utilizado estiver sendo ultrapassado.

Com dificuldades, alunos informaram que estavam levando 9 horas para executar as
tarefas e que ainda não haviam terminado. Porém, eles não buscaram procurar os
documentos disponíveis no site e nem auxílio dos ministrantes.

Vendo estas dificuldades, no próximo encontro buscaram coletar mais informações
sobre os problemas encontrados e criaram um ponto de discussão sobre
características mais simples e separadas da linguagem, como estruturação geral,
ambientes e pacotes. Ao final deste encontro, buscaram unir todas as informações
criando assim um documento final. A partir do terceiro encontro, o foco de
utilização foi direcionado para estruturas matemáticas e equações. No quinto
encontro trabalhou-se a inclusão de gráficos em \LaTeX{} e o sexto buscou-se
incluir customizações em comandos e bibliografias. No último encontro, foi
apresentado o pacote da ferramenta para construção de Teses no formato utilizado
pela Universidade.

Durante as aulas, a utilização livre do acesso à Internet foi notado como um
problema para o desvio de atenção dos alunos. Após solicitarem aos alunos que
não utilizassem tais elementos não pertencentes à aula, houve um maior respeito
com os ministrantes.

\subsection{Retornos}

Ao final do curso, são apresentados questionários respondidos anonimamente pelos
alunos, coletados com informações sobre as aulas. Nestes, foram encontrados
comentários sobre a utilização, enfatizando a utilização do site, a apresentação
da ferramenta utilizando exemplos, curso assumindo que os alunos não possuem
conhecimentos da ferramenta, alternativa aos editores de texto mais populares e
que ficaram felizes em aprender algo que pode ser útil em suas carreiras.

Pontos negativos também foram apresentados, como não instrução de como utilizar
pacotes específicos para \LaTeX{}, reclamações de que o curso possui muitas
tarefas para único crédito e que as tarefas extraclasse levam muito tempo. Com
base nestas informações, sugestões foram apresentas, visando aumentar a carga
horária para 12 encontros ou um semestre completo e diminuição de tarefas
extraclasse que deveria ser disponibilizada mais gradualmente.

Com base nestas informações, concluiu-se que deveriam aumentar a carga horária
do curso, tendo em vista que as aulas foram ministradas de uma forma muito
rápida. A quantidade de alunos deveria ser diminuída e possuir mais encontros
semanais, buscando assim melhorar o atendimento individual. Além disso, deveriam
auxiliar mais os alunos que possuem dificuldades na utilização do \LaTeX{} em
seus próprios computadores.

Concluindo as aplicações, apresentaram que o curso ministrado deveria ser
considerado como extra curricular. Consultas posteriores informam que cerca de
metade dos alunos que assistiram ao curso ainda continuam a utilizar a
ferramenta, caracterizando assim um entusiasmo por parte deles.

% Conversor --------------------------------------------------------------------
\section{Conversor}
\label{sec:conve}

Buscando utilizar um sistema colaborativo para anotações em palestras e aulas
expositivas, criou-se um conversor de conteúdo \LaTeX{} para Wiki
\cite{oneill2005}. Utilizando uma plataforma \textit{web}, adiciona-se uma
palestra em formato \LaTeX{} ao sistema e este pode gerar os \textit{slides} das
apresentações e uma página que pode ser editada no formato Wiki pelos alunos,
visando inclusão de anotações feitas pelos ministrados.

Para facilitar o desenvolvimento dos \textit{slides} das palestras, criou-se um
conjunto de macros em \LaTeX{} que facilitam a produção dos documentos, visando
facilitar a tradução do conteúdo para o formato Wiki. O sistema então cria uma
página no seu sistema que pode ser editada pelos alunos utilizando como entrada
o formato Wiki.

Esta técnica foi utilizada em muitas aulas na instituição, incluindo aulas de
estruturas de dados, sistemas operacionais e linguagens de programação. O único
problema encontrado é que os alunos necessitam possuir responsabilidades sobre o
conteúdo e, em alguns casos, integrantes específicos da turma são selecionados
para cuidar do conteúdo que está sendo modificado pelo grupo, trabalhando assim
como moderadores. Outro problema encontrado durante a tradução é que a
utilização de fórmulas para as páginas do sistema não podem ser tão bem
apresentadas como nos \textit{slides}, porque estas são apresentadas utilizando
HTML.

Não foram encontrados relatos sobre aceitação dos alunos pelo formato Wiki
apresentado e nem o porquê da utilização desta linguagem e não diretamente em
\LaTeX{}.

% Padronização -----------------------------------------------------------------
\section{Padronização}
\label{sec:padro}

Os sistemas de conteúdo comunitário, onde uma pessoa pode alterar o seu conteúdo
utilizando determinado tipo de marcação de texto, é chamado de Wiki
\cite{junghans20081}. A padronização desta linguagem de marcação de textos é
construída utilizando prosa, dificultando a construção e podendo gerar uma
implementação do conteúdo de forma errada.

Motores para renderização deste formato não possuem uma padronização de
linguagem formalizado, onde usuários necessitam aprender marcações diferentes de
texto dependendo do sistema utilizado. Para tanto, utilizando o formato Wiki
Creole 1.0, definiu-se uma gramática BNF que busca especificar a estrutura da
linguagem que deve ser utilizada \cite{junghans20081,junghans20082}.

% Conclusão --------------------------------------------------------------------
\section{Conclusão}
\label{sec:concl}

Utilizando as informações apresentadas por autores, podemos concluir que o
sistema \LaTeX{} pode ser considerado ruim quando o assunto é o aprendizado,
fazendo com que o aluno perca muitas horas resolvendo um simples problema
inicial \cite{gray2003} ou a exibição de erros não descreve de forma clara os
problemas gerados durante a tradução do arquivo de entrada \cite{ha2007}.

Autor também afirma que a curva de aprendizado inicial é considerada íngreme
para o \LaTeX{} \cite{salzberg2005}. Por outro lado, existe uma aplicação que
busca apresentar uma linguagem de marcação de texto mais simples, como a Wiki
Creole, para utilização de alunos e que é traduzida para \LaTeX{}, gerando
apresentações para palestras \cite{oneill2005}. Esta linguagem também possui um
outro fator de utilização que ela possui uma padronização na gramática que a
gera, facilitando ainda mais o desenvolvimento de um tradutor entre a linguagem
Wiki e \LaTeX{}.

A pesquisa de assuntos sobre facilitar a escrita de documentos no formato
\LaTeX{} utilizando uma linguagem de programação mais simples pode ser
considerada difícil, pois não foram encontrados nenhum estudo na área mais
aprofundado. Esta busca de estudos relacionados pode estar sendo executada
utilizando informações erradas, porém este é um fator que ainda será dialogado.

A construção de um sistema que recebe como parâmetro um conteúdo representante
para artigos científicos numa linguagem de mais alto nível se comparada com
\LaTeX{}, como a Wiki Creole, pode ser utilizada como linguagem fonte para
geração de saída alvo \LaTeX{} e posterior compilação deste como artigo
científico, facilitando o desenvolvimento destes documentos pelo usuário, alunos
da graduação; verificar o porquê da não utilização da linguagem alvo também é um
ponto interessante de estudo, principalmente para facilitar a construção do
sistema gerador de artigos científicos.

% Bibliografia -----------------------------------------------------------------
\bibliographystyle{sbc}
\bibliography{document}

\end{document}

